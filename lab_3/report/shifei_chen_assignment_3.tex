%%%%%%%%%%%%%%%%%%%%%%%%%%%%%%%%%%%%%%%%%%%%%%%%%%%%%%%
%%% LATEX FORMATTING - LEAVE AS IS %%%%%%%%%%%%%%%%%%%%
\documentclass[11pt]{article} % documenttype: article
\usepackage[top=20mm,left=20mm,right=20mm,bottom=15mm,headsep=15pt,footskip=15pt,a4paper]{geometry} % customize margins
\usepackage{times} % fonttype
\usepackage{url}
\makeatletter         
\def\@maketitle{   % custom maketitle 
\begin{center}
{\bfseries \@title}
{\bfseries \@author}
\end{center}
\smallskip \hrule \bigskip }

%%%%%%%%%%%%%%%%%%%%%%%%%%%%%%%%%%%%%%%%%%%%%%%%%%%%%%%%%%%%%%%%%%%%
%%% MAKE CHANGES HERE %%%%%%%%%%%%%%%%%%%%%%%%%%%%%%%%%%%%%%%%%%%%%%
\title{{\LARGE Machine Learning in Natural Language Processing: \newline Assignment 3}\\[1.5mm]} % Replace 'X' by number of Assignment
\author{Shifei Chen} % Replace 'Firstname Lastname' by your name.

%%%%%%%%%%%%%%%%%%%%%%%%%%%%%%%%%%%%%%%%%%%%%%%%%%%%%%%%%%%%%%%%%%%%
%%% BEGIN DOCUMENT %%%%%%%%%%%%%%%%%%%%%%%%%%%%%%%%%%%%%%%%%%%%%%%%%
%%% From here on, edit document. Use sections, subsections, etc.
%%% to structure your answers.
\begin{document}
\maketitle

\section{Implementation}
In the official PyTorch document\footnote{\url{https://pytorch.org/docs/stable/optim.html}} there is a tutorial on how to use an optimizer. So following the tutorial, we just need to create an optimizer and plug in its parameters first (SGD in this case as it is required in the lab instruction, also remember to put in \verb|weight_decay| as it is useful in the later work), clear the gradients by calling \verb|optimizer.zero_grads()|, calculate the loss gradients by \verb|training_loss.backward()|, step forward (\verb|oprimizer.step()|) and we will make the gradient decent.

\end{document}